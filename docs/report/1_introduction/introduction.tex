\chapter{Introduction}
\label{chap:intro}
\textbf{Version Control} is a critical aspect of software development that helps developers keep track of changes made to a codebase. It enables software teams to work collaboratively, efficiently, and accurately on projects, reducing the likelihood of errors and conflicts. In essence, version control is the process of tracking and managing changes to files over time.
\vspace{9pt}

\textbf{Version Control System (VCS)} is a software tool that automates the version control process. It provides a centralised repository where developers can store their code, track changes, and collaborate with other team members. The Version Control System ensures that each team member has access to the latest version of the code and can work on it simultaneously.
\vspace{9pt}

The impact of \textbf{Version Control Systems} on software development has been immense. Before the advent of Version Control Systems, developers used to rely on manual processes to track changes, which was time-consuming and error-prone. With VCS, software teams can work together more efficiently, manage changes more effectively, and deliver better-quality software products.
\vspace{9pt}

Version Control Systems have also facilitated the rapid growth of \textbf{Continuous Integration} and \textbf{Continuous Delivery} (CI/CD), which have become essential systems in modern large scale software development. Overall, Version Control Systems have revolutionised how software is developed, making it easier, faster, and more reliable.

\section{Motivation}
\noindent
Over the years, Version Control Systems have become essential for software developers, enabling them to collaborate and work more effectively. As the software development industry has evolved, so have Version Control Systems, leading to the creation of many different software solutions, each with unique features and strengths. However, these systems' core data structures and algorithms have remained relatively unchanged.
\vspace{9pt}

The performance of Version Control Systems can be crucial in determining the efficiency of software development projects. Slow or inefficient Version Control Systems can result in delays, errors, and even project failure. Therefore, it is important to explore alternative data structures and algorithms that could improve the performance of these systems.

\section{Objective}
\noindent
Version Control Systems (VCS) have become essential for software developers to collaborate effectively and efficiently. However, many users may not fully comprehend the intricacies of the underlying concepts at the core of these systems.
\vspace{9pt}

This report aimed to address this knowledge gap by exploring the underlying data structures and algorithms used to power Version Control Systems and by evaluating potential trade-offs and benefits of alternative approaches. In order to provide insight into how these factors impact system performance, scalability, and overall effectiveness.
\vspace{9pt}

In addition, this report aimed to provide a comprehensive overview of the evolution of Version Control Systems, including an overview of the most popular version control solutions throughout the years, such as \nameref{sec:sccs}|1972, \nameref{sec:svn}|2000, and \nameref{sec:git}|2005.