\chapter{Background}
% ==============================================================================
\section{What is a Version Control System}
% Version Control Systems (VCS) are systems that store the history of a set of files. They track modifications made to files by individual users and allow users to revert files to previous versions. While VCSs can be used to track changes in any set of files, they are commonly used to track changes in source code. The use of VCSs in the software industry has become crucial for the development of large software projects which require developer collaboration. The use of VCSs in the software industry has become so common that many software development companies have adopted their own VCSs.
% VCSs are also known as revision control systems, revision management systems, source control systems, and source code management systems. This document will use the term VCS to refer to any of these systems.
% TODO: OR
A version control system saves modifications done by individual software developers which makes the process easier for them since they can track their work over time. It helps share data between nodes where each node can be kept up to date with updated versions so there is no need for any merge conflicts. The advantages provided by VCS include: it aids in collaboration among programmers, improves efficiency when working on larger groups of mixed products; provides an audit trail; easy branching functionality; simplifies team-based development; understanding who has worked on specific pieces or sections during what period in time but also making sure that all changes are tracked properly providing transparency within teams while maintaining optimal performance levels through streamlined management processes, helping avoid errors arising from inconsistency between versions such as mismatched documents or programming problems resulting from conflicting edit operations
\subsection{What is the purpose of Version Control Systems}
% TODO: Explain the purpose of VCSs
The main purpose of VCSs is collaboration...
\section{Types of VCS}
Over the years, two approaches to VCSs have emerged - Centralized and Distributed. Both approaches are in widespread use today. The centralized approach is based on the client-server model where there is a single central repository that stores the history of all files, while the distributed approach provides each user with a full copy of the repository.
The core difference between the two approaches is that the centralized approach requires a single point of failure, while the distributed approach does not. The centralized approach is also more difficult to scale than the distributed approach.
\subsection{Centralized Version Control Systems (CVCS)}
% TODO: Explain centralized VCS
A centralized Version Control System is a system that enables developers to work together on the same project by storing the main copy of files in a central repository. This system keeps track of all files and saves information in the local repository. CVCS are called centralized because there is only one central server or repository. The server maintains a complete record of issues, while clients only maintain a local copy of the shared documents. All developers make their modifications on the repository through checkout but only the last version of the files is retrieved from the server, which means that any modifications made will be automatically shared with other developers. Users can modify in parallel with their local copy of shared documents and sync with the central server to release their contributions and make them visible to other collaborators. Because centralized version control systems rely on one repository that includes the correct version of the project, it must restrict write accesses so that only trusted contributors are allowed to commit modifications. CVCS has some challenges, such as if the central server is inaccessible, then users will not be able to merge their work at all or save the released modifications. It is also if the central repository is corrupted, everything will be lost. Contributors must be the ones who have writing permissions to perform basic tasks, such as reverting modifications to a previous state, creating or merging branches, or releasing modifications with full revision history. This limitation affects participation and authorship for new contributors. So, the main drawbacks of using CVCS are that it requires a network connection to work on the source code, developers must order to contribute to a project, and a single point of failure is an issue when using one server.

\subsection{Distributed Version Control Systems (DVCS)}
% TODO: Explain distributed VCS
% There are many advantages to using a distributed version control system (DVCS) over a centralized version control system (CVCS). DVCS is designed to work in two ways: it keeps the entire file history on each device locally and can also sync local modifications the user made with the server again when necessary so that the modifications can be shared with the whole team. In DVCS, the developers can work with different groups of people in different ways working on the same project. As well as any repository can be cloned, and, from a conceptual point of view, no repository is more significant than any other.
% In practice, the development team will organize the repositories in the hierarchy and at least one of the repositories will be marked as the central repository. To provide a new method for versioning software artifacts, several Distributed Version Control Systems emerged in the software field, such as Mercurial, Git, and Bazaar. These tools have been adopted by many Open Source Software (OSS) projects.
% The operations in DVCS are much faster than operations in CVCS because they are local. DVCS is considered to be the future of version control systems because it is suited for huge projects with more independent developers, and provides important advantages by allowing users to work and use a complete version control feature set even when there is no network connection. DVCS allows for version control of modifications done locally, allowing early drafts of work to be revised without requiring it to be released to others.
% There are many advantages to using DVCS: flexibility, hosting services like Github, availability, it is very fast due to its local nature to the majority of operations, it doesn’t require access to remote servers, and branching and merging can be done very easily in DVCS. Collaboration between team members and allow individual developers to be servers or clients are the most important features offer by version control systems, so developers can work on source code without being connected to a central or remote repository.
% There are some challenges introduced by DVCS: it lacks an understandable version numbering system, where there is no centralized versioning server, and uses hash modifications or a unique GUID. So, the lack of a central server makes system backup so difficult. The two most popular complaints about the disadvantages of DVCS are that: pessimistic locks are not available, and they have weak tools for binary.
% The reasons for the transition from centralized to decentralized version Control Systems are the ability to work offline and the ability to work incrementally. The ability of developers to made several roles, such as developing a new task or fixing errors, and the ability to do exploratory coding efficiently.

Distributed Version Control Systems (DVCS) were created to overcome the limitations of Centralized Version Control Systems, which enable branching and merging, avoid local VCS operations and allow developers collaboration.
Because of the limitations in using a centralized version control system, Open Source Software (OSS) projects today largely adopt DVCS. DVCS is designed to work in two ways: it keeps entire file histories locally on each device and can also sync local modifications made by users with servers again when necessary so that these modifications can be shared with everyone else. In DVCS developers are able to work separately or together but working on the same project as they have access to all repositories needed for their task; any repository can be cloned from another one so there's no more important repository than others.
To provide a new way for versioning software artifacts several Distributed Version Control Systems emerged in the software field such as Mercurial Git Bazaar etc., these tools have been adopted by many Open Source Software (OSS). The operations in DVCS are much faster than those found in CVSS because they're done locally while CVSS operations require remote connection; some consider that distributed systems will soon replace centralized ones because they suit bigger projects with more independent developers who want full functionality even without network connection available, offer advantages like earlier drafts of your work being saved without requiring you releasing them publicly or sharing them with other people etc.; three major advantages offered by use of DVSs include flexibility availability which is related solely to its ability not needing access remotely servers for most operation types plus it's very fast due to its locality-based nature--most actions only need occur at one location unlike CVSCS versions where changes must travel through an often slower internet link before reaching their destination; branching merging is easy when working within DVSs due both high level abstraction language features provided but also lack requirement that user has knowledge about how specific models like Subversion operates making this process easier then would otherwise be possible under traditional model; collaboration between members/individual developer vs server/client differences make up most important feature offered by VSCs since team member may work on source code regardless whether connected centrally or remotely--this allows individual developer do exploratory coding efficiently too. There are challenges introduced by use if DVSs including system backup being difficult cause lack central server makes creating backup virtually impossible meanwhile Pessimistic Locks aren't available either nor does present day software offer good support for binary files. Reasons why transition was made from centralized systems towards decentralized ones: offline capability enabling development portion done independently while maintaining integration into main branch + incremental capabilities providing greater efficiency during exploratory coding efforts.
% ==============================================================================
\section{Existing Version Control Systems}
\subsection{Local Version Control Systems}
Early VCS were intended to track changes for individual files and checked-out files could only be edited locally by one user at a time. They were built on the assumption that all users would log into the same shared Unix host with their own accounts. As you can imagine, these early systems made it easier for small teams to revisit code states from various points in history.
\subsubsection{Source Code Control System (SCCS)}
% TODO: Explain SCCS
SCCS was released in 1972, and it is one of the first successful VCS tools. It was written by Marc Rochkind at Bell Labs, who wanted to solve the problem of tracking file revisions. The tool made it significantly easier to track down bugs introduced into a program. SCCS is worth understanding at a basic level because it helped set up modern VCS tools that developers use today.
\paragraph{Architecture}
% TODO: Rewrite this section
\subsubsection{Revision Control System (RCS)}
% TODO: Explain RCS
\paragraph{Advantages}
% TODO: Add advantages
\paragraph{Disadvantages}
% TODO: Add disadvantages
\subsection{Centralized Version Control Systems}
Version control technology continued to evolve, leading to centralized repositories that contained the 'official' versions of their projects. This was good progress, since it allowed multiple users to checkout and work with the code at the same time as well as making commits back into this central repository. Furthermore, network access was required for people who wanted to commit changes they had made locally.
\subsubsection{Concurrent Versions System (CVS)}
Concurrent Versions System (CVS) was developed in the 1980s by Larry Wall and was the first VCS to become widely used for collaborative software development. Concurrent Versions System is a centralized VCS, meaning that there is a single server that stores the entire history of the project. The server is the only place where files can be added, removed or modified.
\paragraph{Advantages}
% TODO: Add advantages
\paragraph{Disadvantages}
% TODO: Add disadvantages
\subsubsection{Apache Subversion (SVN)}
Subversion (SVN) is a centralized VCS that was developed by CollabNet in the year 2000. Subversion was developed to replace CVS, which was becoming increasingly difficult to maintain.
\paragraph{Advantages}
% TODO: Add advantages
\paragraph{Disadvantages}
% TODO: Add disadvantages
\subsubsection{Perforce}
Perforce is a centralized VCS that was developed by Perforce Software in 1995. Perforce is a commercial VCS that is used by many large companies, such as Google, Adobe and IBM. It is still one of the largest version control systems in use today.
\paragraph{Advantages}
% TODO: Add advantages
\paragraph{Disadvantages}
% TODO: Add disadvantages
\subsection{Distributed Version Control Systems}
In a distributed version control system, all copies of the repository are created equal - there is no central copy of the repository. This design principle encourages commits, branches and merges to be created locally without network access and pushed to other repositories as needed.
\subsubsection{Git}
Git is a distributed VCS that was developed by Linus Torvalds in 2005. Git is a free and open source VCS that is used by many large companies, such as Google, Facebook and Twitter. It is one of the most popular version control systems in use today.
\paragraph{Advantages}
% TODO: Add advantages
\paragraph{Disadvantages}
% TODO: Add disadvantages
\subsubsection{Mercurial}
Mercurial is a distributed VCS that was developed by Matt Mackall in 2005. Mercurial was developed with the same goal as Git, to maintain the Linux kernal project.
\paragraph{Advantages}
% TODO: Add advantages
\paragraph{Disadvantages}
% TODO: Add disadvantages
\subsubsection{BitKeeper}
% TODO: Explain BitKeeper
\paragraph{Advantages}
% TODO: Add advantages
\paragraph{Disadvantages}
% TODO: Add disadvantages
\subsubsection{Darcs Advanced Revision Control System (Darcs)}
% TODO: Explain Darcs
\paragraph{Advantages}
% TODO: Add advantages
\paragraph{Disadvantages}
% TODO: Add disadvantages
\subsubsection{Monotone}
% TODO: Explain Monotone
\paragraph{Advantages}
% TODO: Add advantages
\paragraph{Disadvantages}
% TODO: Add disadvantages
\subsubsection{Bazaar}
% TODO: Explain Bazaar
\paragraph{Advantages}
% TODO: Add advantages
\paragraph{Disadvantages}
% TODO: Add disadvantages
\subsubsection{Fossil}
% TODO: Explain Fossil
\paragraph{Advantages}
% TODO: Add advantages
\paragraph{Disadvantages}
% TODO: Add disadvantages
% ==============================================================================
\section{Summary of Key Features}
\subsection{Repositories}
% TODO: Explain repositories
\subsection{Commits}
% TODO: Explain commits
\subsection{Branching and Merging}
% TODO: Explain branching and merging
\subsection{Pulling and Pushing}
% TODO: Explain pulling and pushing
